In many machine learning applications, the observed data is a
discretely-sampled realisation of a continuous process. While one can
naively treat such data as vectors in Euclidean space, one may do better
to regard them as functional data.  
When the underlying curves of interest lie on a manifold, e.g.~probability density
functions or warped curves of a common template function, there is further structure to exploit. 
This work addresses the estimation of pairwise geodesic
distances between functional manifold data that are observed with noise,
causing them to lie off the manifold. This setting falls outside of
classic manifold learning techniques which require data to live exactly
on or very near a manifold. The proposed methodology first sends the
observed functional data to the hidden manifold, estimated using
subspace-constrained mean-shift. Geodesic distances are subsequently
calculated by employing shortest-path algorithms on this estimated
manifold. Improved estimation of the pairwise geodesic distance has
potential benefits for downstream tasks such as distance-based functional classification.